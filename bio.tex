%; /usr/bin/env mkdir -p .build && xelatex -output-directory=.build -halt-on-error $0 && cp .build/*.pdf ./; exit

% Add 'print' as an option in the square bracket to remove colors from this template for printing
\documentclass[]{friggeri-cv}
\begin{document}

\header{Mocy}{Sheng}{Software Designer}

% ----------------------------------------------------------------------------------------
% SIDEBAR SECTION
% ----------------------------------------------------------------------------------------

\begin{aside} % In the aside, each new line forces a line break
  \section{Contact}
  Mobile:
  +1 (415) 741--7785
  ~
  \href{mailto:ranmocy+cv@gmail.com}{ranmocy@gmail.com}
  \href{https://ranmocy.me}{https://ranmocy.me}
  \href{https://www.linkedin.com/in/ranmocy}{ranmocy@linkedin}
  \href{https://github.com/ranmocy}{ranmocy@github}
  \href{https://twitter.com/ranmocy}{ranmocy@X}
\end{aside}

From leading engineering teams at startups to driving large-scale innovations at tech giants, 
Mocy Sheng's career is a testament to his expertise in software system architecture and machine learning.
His contributions continue to shape the future of technology, 
whether in smart contract security, mobile infrastructure, or AI-driven applications. 
With a relentless drive for innovation and a keen eye for efficiency, 
he stands at the forefront of modern software engineering.

% ----------------------------------------------------------------------------------------
% WORK EXPERIENCE SECTION
% ----------------------------------------------------------------------------------------

\section{Experience}

\textbf{Apple \emph{--- Senior Software Engineer}}

Mocy currently serves as a Senior Software Engineer at Apple, contributing to the Apple Vision Pro Services team.
His work focuses on developing cloud and machine learning services that enhance visionOS features,
leveraging his expertise in large-scale software architecture and AI-driven solutions.

\textbf{Narya.ai \emph{--- Head of Engineering / Founding Engineer}}

Before Apple, Mocy was the Head of Engineering and a founding engineer at Narya AI.
From 2022 to 2024, he led the company's ambitious mission to revolutionize smart contract security.
He owned the entire product spectrum - from software engineering and machine learning to product strategy and UI — 
guiding a team of five in building a no-code platform for testing and securing EVM-based smart contracts.

At Narya AI, Mocy led the development of a proprietary Rust-based fuzzing engine,
an innovation that dramatically outperformed traditional smart contract audit tools.
The engine identified vulnerabilities faster and more effectively than competitors like Foundry or Echidna,
reducing the reliance on manual audits. 
Furthermore, he integrated LLM-based code generation, 
allowing NFT projects to automate test creation in minutes rather than days, 
a breakthrough that saved engineers countless hours while improving reliability and security.

\textbf{Google \emph{--- Senior Software Engineer}}

Mocy spent seven years at Google from 2015 to 2022, 
where he became a key figure in the evolution of Google Play Services and Family Link. 
He was instrumental in maintaining and improving the infrastructure that runs on over 2 billion Android devices, 
collaborating across 100+ module teams to enhance software reliability and developer experience.

One of his significant contributions was revamping Android components virtualization through code generation, 
simplifying maintenance for developers while ensuring backward compatibility. 
He also championed the adoption of Kotlin across multiple teams, 
streamlining development and improving code efficiency.

Within Google's Family Link team, Mocy led efforts to enhance parental supervision tools for millions of users.
He reworked the complex Time Limit feature, reducing user complaints by 90\%. 
Additionally, he designed a policy-based framework that improved feature reliability 
and established robust logging and A/B testing mechanisms, ensuring seamless and error-free releases.

\textbf{Early career}

Mocy's journey into software engineering began with his role as a full-stack developer at GitCafe from 2011 to 2013,
a Git hosting platform based in Shanghai. 
He played a pivotal role in designing the service's core infrastructure,
from Git servers to billing management, and even led efforts to optimize performance and 
implement a metrics system that improved user experience. 

Following this, he took on a summer internship at Twitter Ads in 2014, 
where he worked on full-stack development using Ruby on Rails and Backbone.js. 
His contributions included optimizing campaign performance and introducing a tweet preview feature for advertisers.

% ----------------------------------------------------------------------------------------
% EDUCATION SECTION
% ----------------------------------------------------------------------------------------

\section{Education}

\begin{entrylist}
  % ------------------------------------------------
  \entry{2013 --- 2015}
  {Master \textnormal{of Computer Science}}
  {University of San Francisco}
  {
    Algorithms, Artificial Intelligence, Theory of Computation
  }
  % ------------------------------------------------
  \entry{2009 --- 2013}
  {Bachelor \textnormal{of Mechanical Design}}
  {Shanghai Jiao Tong University}
  {
    Robotics, Design and Manufacture, Electromechanical Control, Finite Element Method
  }
  % ------------------------------------------------
\end{entrylist}

His technical prowess was evident early on—he secured the First Prize in China’s National Olympiad in Informatics in Provinces (NOIP) for three consecutive years (2006–2008), an achievement that underscored his deep understanding of algorithms and problem-solving.

% ----------------------------------------------------------------------------------------
% TOY SECTION
% ----------------------------------------------------------------------------------------

\section{Beyond Work}

Beyond his professional achievements, Mocy has a passion for building innovative side projects. 
He developed Idid.Im, a React-based app designed to help users organize their lives, 
and Monkey Tree, an Android app that fixes phonetic name sorting for Chinese contacts, 
earning a 4.8/5 rating on the Play Store. 
His contributions to open-source projects, including SparkDriverLess and Web-Stub, 
further showcase his commitment to pushing the boundaries of software development.

\end{document}
